\documentclass{book}

\usepackage{setspace}
\usepackage{paralist}
\usepackage{multicol} 
\usepackage{hyperref}
\usepackage{shapepar}
\usepackage{amssymb}
\usepackage{breqn}
\usepackage{amsfonts}
\usepackage{subfigure}
\usepackage{amsmath, amsthm, amscd}
\usepackage{caption}
\usepackage{bidipoem}
\usepackage[all]{xy}
\usepackage{makeidx}
\usepackage{tikz,pgf,pgfplots}
%\usepackage{xindy,acronym,nonumberlist=true}{glossaries}
\usepackage{xepersian}
\makeindex

\settextfont{Traditional Arabic}
\setcounter{tocdepth}{4}
\setcounter{secnumdepth}{4}
\setdigitfont{Traditional Arabic}


\newtheorem{thm}{قضیه}[chapter]
\newtheorem{Def}{تعریف}[section]

\newcommand{\dd}{\, \mathrm{d}}
\newcommand{\mian}[2]{\frac{#1_1 + #1_2 + \dots + #1_{#2}}{n}}

\defpersianfont\BNaz{B NAZANIN.TTF}
\deflatinfont\Time{TIMES.TTF}

\huge{\title{\lr{$5th$ $File$}}}
\author{علی}

\begin{document}
	\maketitle
	%...
	%\doublespacing
	\tableofcontents
	\singlespacing
	%...
	\listoffigures
	\listoftables
	\Time
\chapter{\lr{Code Insertion}}
\begin{latin}
	\begin{verbatim}
		close all; clear all; clc
		A = 4; fm = 1; M = 20;
		t = 0:(1/(M*fm)):1;
		xs = A*cos(2*pi*fm*t);  %sampled cosine signal
		%Quantization
		L = 10;
		Length_xs = length(xs);
		xq = zeros(1, Length_xs);
		L_level = zeros(1, Length_xs);
		for loop = 1:Length_xs
		for index = 1:L
		if (A-index*(2*A)/L <=xs(loop)) && (A-(index-1)*(2*A)/L > xs(loop)) 
		xq(loop) = A - index*(2*A)/L + A/L;
		L_level(loop) = L-(index-1); %-------------------------------------Z1
		end
		end
		if xs(loop) == A
		xq(loop) = A - A/L;
		L_level(loop) = L;  %-----------------------------------------Z2
		end
		end
		%xq is the quantized cosine signal
		NN = ceil(log2(L));  %---------------------------------------------Z3
		encoded = dec2bin(L_level, NN) %-----------------------------------Z4
		plot(t, xs, 'r-')
		hold on
		plot(t, xq, 'b')
		xlabel('time')
		ylabel('signals')
		legend('Original signal','quantized signal')
		
		%% 
		clc;
		t = [0:.1:2*pi]; % Times at which to sample the sine function
		sig = sin(t); % Original signal, a sine wave
		partition = [-1:.2:1]; % Length 11, to represent 12 intervals
		codebook = [-1.2:.2:1]; % Length 12, one entry for each interval
		[index,quants] = quantiz(sig,partition,codebook); % Quantize.
		plot(t,sig,'x',t,quants,'.')
		grid on;
		legend('Original signal','Quantized signal');
		axis([-.2 7 -1.2 1.2])
		%%
		clear,
		clc;
		pdf='exp(-(x-m).^2/2/sigma^2)/sqrt(2*pi)/sigma'; %Gaussian pdf of x
		xf=inline(['x.*' pdf],'x','m','sigma');
		f=inline(pdf,'x','m','sigma');
		m=0; sigma=1; % Mean and variance of the random variable x
		b0=-3; bN=3; % Given least/greatest value of the random variable x
		for N=5:6 % Number of quantization intervals
		delta=(bN-b0)/N; 
		b=b0+[0:N]*delta;
		msqe=0; % Mean-Square Quantization Error
		for i=1:N % Centroid of each interval
		tmp1=quad(xf,b(i),b(i+1),0.01,[],m,sigma);
		tmp2=quad(f,b(i),b(i+1),0.01,[],m,sigma);
		tmp=tmp1/tmp2;
		c(i)=tmp; 
		x2f=inline(['(x-tmp).^2.*' pdf],'x','m','sigma','tmp');
		msqe = msqe + quad(x2f,b(i),b(i+1),0.01,[],m,sigma,tmp); 
		end
		% Resulting boundary vector and centroid vector
		x=b0+(bN-b0)/1000*[0:1000];
		y(find(x<b(1)))=c(1); % Left-most interval
		for i=1:N
		y(find(b(i)<=x&x<b(i+1)))=c(i);
		end
		y(find(x>=b(N+1)))=c(N); % Right-most interval
		subplot(2,2,N-4), 
		plot(x,y),
		hold on,
		grid on % Quantization graph
		m=0; sigma=1; fx=feval(f,x,m,sigma);
		plot(x,fx,'r:') ;
		% Resulting MSQE
		end
	\end{verbatim}
\end{latin}

\chapter{$Tikz$}
\BNaz
\begin{tikzpicture}
	\draw[color=red](0,0)--(5,5);
	\draw[line width=2pt] (5,5)--(12,2);
\end{tikzpicture}

\begin{figure}
	\label{grid1}
	\caption{\lr{Grid}}
	\begin{center}
		\begin{tikzpicture}
			\draw [step=0.5, color=black, scale=1.1] (-3,-4) grid (3,4);
			\coordinate (P1) at (-3,-3);
			\fill[color=red] (P1) circle (3pt);
			\coordinate (P2) at (3,4);
			\fill[color=blue] (P2) circle (2pt);
			\draw [color=brown, <->, line width=1.25pt] (P1) -- (P2);
		\end{tikzpicture}
	\end{center}
\end{figure}

\begin{figure}
	\begin{center}
		\begin{tikzpicture}
			\draw[line width=1.25pt, color=magenta] (0,0) ellipse (2cm and 1cm);
		\end{tikzpicture}
	\end{center}
	\label{grid2}
	\caption{\lr{Ellipse}}
\end{figure}

\begin{figure}
\begin{center}
	\begin{tikzpicture}
		\begin{axis}[xmin=-7,xmax=7,ymin=-45,ymax=45,
			ylabel=$y$,
			xlabel=$x$, %axis lines
			title={\lr{X^2 in two shapes}},
			legend pos=north west]
			\addplot[domain=-5:5,blue,thick,samples=100]{x^2};
			\addplot[domain=-5:5,cyan,ultra thick, mark=*,samples=50]{-x^2};
			\addplot[green, ultra thick] {sin(deg(x))};
			\addlegendentry{X^2}
			\addlegendentry{-X^2}
			\addlegendentry{Sin(x)}
		\end{axis}
	\end{tikzpicture}
\end{center}
	\caption{\lr{addplot}}
	\label{fig3}
\end{figure}
\chapter{\lr{Glossary}}

\end{document}