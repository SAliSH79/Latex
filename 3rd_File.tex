\documentclass{book}

\usepackage{setspace}
\usepackage{multicol}
\usepackage{hyperref}
\usepackage{shapepar}
\usepackage{amssymb}
\usepackage{breqn}
\usepackage{amsfonts}
\usepackage{amsmath}
\usepackage{xepersian}

\settextfont{Traditional Arabic}
\setcounter{tocdepth}{4}
\setcounter{secnumdepth}{4}
\setdigitfont{Traditional Arabic}

\defpersianfont\BNaz{B NAZANIN.TTF}
\deflatinfont\Time{TIMES.TTF}

\huge{\title{\lr{$Third$ $File$}}}
\author{علی}

\begin{document}
	\maketitle
	%...
	%\doublespacing
	\tableofcontents
	\singlespacing
	%...
\chapter{1}\label{chap1}
\section{سعدی و شعر دو قسمتی در صفحه}
	یک بیت شعر از گلستان سعدی مهمون ما:
\begin{multicols}{2}
	دریاب کنون که نعمتت هست به دست \\ کین دولت ملک می رود دست به دست
\end{multicols}
\section{فرمول نویسی}
فرمول 1:
\[
a = b - c
\]
فرمول ۲:
\[
\boldsymbol{\Sigma} = 2^{\cosh(x)} - \varPhi*B_{\varLambda}
\]
\subsection{کسر و رادیکال}
کسر و رادیکال: \\
\[
\sqrt[n]{\sqrt[n]{\boldsymbol{\mathbf{x - x^{\cos x}}}}}
\]
\[
\tfrac{1}{2} \frac{a - \cos x}{\sqrt[3/5]{x^{2/3}}}
\]
\subsubsection{سه نقطه}
\[
\mathbf{x} = (x_1, x_2, \dots, x_n)
\]
\subsection{انتگرال نامعین}
\[
\int x^2 - \sqrt[n]{\ln \sin x} \, \mathrm{d}x
\]
\subsection{انتگرال معین دوگانه}
\[
\int_{-\infty}^{0^{+}} \int_{-2}^{3y} \sinh ^2 x \, \mathrm{d}x \, \mathrm{d}y
\]
\subsection{تولید حد}
\[
\lim_{x \to \infty} \frac{\sqrt[n]{\log_4 56}}{\sin x^2 - x^3}
\]
\subsection{نماد های ترکیبی}
\begin{flushright}
میانگین:
\[
\bar{x} = \frac{\sum_{-\infty}^{\varOmega} x_i}{n} 
\]
نماد ها:
\[
\widehat{xyz_n}
\]
\[
\widetilde{y_{23}}
\]
\end{flushright}
\lr{:Over/Underset}
\[
\lim_{x \to \infty} f(x) \overset{\mathrm{Hop}}{=\joinrel=} 1/1
\]
\[
\lim_{x \to \infty} \sin 2x \underset{x \xrightarrow{} 0^{+}}{=} 0
\]
\[
\overset{\ast\ast}{X_{2}}
\]
\[
L_i(x) = \prod_{\substack{j = 0, \\ j \neq i}}^n \frac{x - x_j}{x_i - x_j}
\]

\chapter{فرمول نویسی درون خطی}
\BNaz
بنا بر قضیه 
$
\sin ^{2}x - \ln_b \sqrt[n]{\frac{2}{x_3 - 2x}}  
$
می دانیم.
\\
سپس می دانیم علی رغم تلاش بسیار در 
\begin{equation}\label{eq1}
	\sum_{0}^{\infty} x^2
\end{equation}
داریم می بینیم که در \eqref{eq1}
به این صورت مقدار مجموع مربعات تعریف شده است.
\section{فرمول نویسی دوخطی}

داریم که :
\begin{align} 
	\int_{x = 0^{+}}^{55} \tfrac{1}{b} \prod_{j = 1}^{100} \log_3 \sin 2x \mathrm{d}x & = c \notag \\
	& = d_{2} \label{eq2} \\
	& = \sum_{\substack{j = 0, \\ j \neq i}}^{j >= i} \ln_e \epsilon ^{e} \label{eq3} 
\end{align}
می بینیم که در \eqref{eq3}
با شرایط جدید $LATEX$ آشنا شدیم.

\chapter{\lr{Rest of Mathematical Typing}}
\Time
\lr{
$\mathbf{AA}$ \hfill $\mathbf{SS}$
}
\section{\lr{align-split}}
\begin{equation} \label{eq4}
\begin{split}
	F(x) &= \int_{0^{+}}^{\lim_{x \to 1} F_b} \cos x^2  \, \mathrm{d} x \\
	&= F(a) - F(b) \\
	& \ge \ln 10 \\
	& \le 5\tfrac{2}{5}
\end{split}
\end{equation}
\BNaz
\begin{dgroup}
\begin{dmath}
   \sin ^2 x + \cos ^2 x = 1 \label{I}
\end{dmath}

\begin{dmath}
	\ln e = 1
\end{dmath}
\end{dgroup}
در قضیه \eqref{I}
می بینیم که .....
\begin{align}
	F_2(b) &= 23 \\
	F_3(b) &= 23/3
\end{align}

\[
\mathbb{R, N, Q}
\]
\[
\mathcal{A, G, F, B}
\]








\end{document}