%%%%%%%%% ALi SH %%%%%%%%%%%%%%%%%%

\documentclass[12pt]{book}
\usepackage{hyperref}
\setcounter{secnumdepth}{4}
\setcounter{tocdepth}{4}
\usepackage{multicol}
\usepackage{shapepar}
\usepackage{amsmath}
\usepackage{xepersian}
\settextfont{B Nazanin.ttf}
\setdigitfont{Mj_Diwani Normal.ttf}

\title{متن آزمایشی}
\author{علی حسینی }
%\date{20 آذر 2022} 
%\date{}   No date will shown
\defpersianfont\nast{IranNastaliq.ttf}

\begin{document}
%\lr {Hello, This is my first latext practice.}
\maketitle
\chapter{مقدمات}\label{chap1}
	سلام این اولین فایل لاتک من است.
	  بی تو مهتاب شبی باز از آن کوچه بگذشتم! همه تن چشم شدم خیره به دنبال تو گشتم. شوق دیدار تو لبریز شد از جام وجودم.  \\ بنابر قضیه گرین\LTRfootnote{Green} داریم : ....
	 
	  همچنین در قضیه مقدار میانگین\LTRfootnote{Mean Value Theorem} داریم ...
	  همچنین برای قضیه $X$ \RTLfootnote{که در سال ۱۹۹۴ توسط شخصی در آلمان تست شد و در سوییس به ثبت رسید} داریم .....
%{\Huge پاراگراف بعدی}
%{\tiny پاراگراف بعدی}
%{\normalsize پاراگراف بعدی}
	%می شود

%	می‌شود %% enter + enter
\textrm{متن معمولی} 
% حالت معمولی

{\Huge \nast  متن نستعلیق با فونت متفاوت از کل متن }

\textbf{متن پررنگ}

\textit{متن خوابیده}

\underline{متن زیرخط دار}

\fbox{متن درون جعبه}

\emph{متن تاکید شده}

%%%%%%%%%%%%%%%%%%Centering%%%%%%%%%%%%%%%%
\chapter{وسط چینی}\label{chap2}
	بعد از مطالعه فصل \ref{chap1} حال به سراغ درس بعدی می‌رویم:
\section{وسط‌چینی}
\subsection{دستور $Center$}
\subsubsection{$Center$}
\begin{center}

	
	این متن در وسط نوشته شده است با دستور 
	\lr{begin center}
\end{center}
\subsubsection{$Center line$}
\centerline{متن در وسط با دستور \lr {center line}}


\subsection{دستور $Flush$}
\begin{flushleft}
	متن \textit{چپ‌چین} شده
\end{flushleft}

\begin{flushright}
	\textbf{متن} راست‌چین شده
\end{flushright}

به نام حق \\
جلسه اول از آموزش یادگیری $LATEX$ از آموزش دکتر مس‌فروش

\subsection{دستور $Centering$}
\centering
متن در وسط نوشته شده با دستور \lr {centering}

%\today
\chapter{حروف چینی}\label{chap3}
\begin{multicols}{2}
	
	\shapepar{\hexagonshape}
سلام من علی هستم بیست و دو ساله تهران دارم بهترین ها برای شماست. موفق باشید به امید دیدار شما در آینده نزدیک از صمیم قلب سلامتی براتون میخوام
\end{multicols}
\chapter{فرمول نویسی}\label{chap4}

$28 /march/ 2023$
\[
a = b + c \qquad \qquad d  = e - f 
\]
\[
\Sigma \qquad \kappa \qquad \Phi \qquad \eta \qquad \Xi
\]
\[
\boldsymbol{\varOmega, \varDelta}
\]
\\
\[
\alpha^{2\varGamma} - 2_{\beta} = 5.5^{23}
\]
\[
\mathbf{x} = (x_1,x_2,x_3)
\]
\[
\huge{\sqrt[\frac72]{x^{\varPhi^{2}} - y^\varOmega}}
\]
\[
\frac{x^2 - x^4}{\sqrt[3]{x^\varphi - 2.345}}
\]
\[
A \cdot B \qquad \mathbf{x} = (x_1,x_2,\dots,x_n)
\]
\[
\int (x^2 - 2x + 3) \, \mathrm{d}x
\]
\[
\int_{\tfrac{\pi}2}^{2} \sinh(x) \, \mathrm{d}x
\]
\[
\idotsint sin(x+y) \, \mathrm{d}x \dots \mathrm{d}y
\]
\[
\int_{-2}^{3} \int_{\tfrac{\pi}{3}}^{3} sin(x^y) \mathrm{d}x \mathrm{d}y
\]

\[
\int_{0}^{\infty} \frac{2\ln x - 5\sin^3 x}{\sqrt[3]{5 + x_3 - \sec x^{4+j}}} \, \mathrm{d}x
\]
\[
\lim f(x) 
\]
\[
\lim_{x \to 0} \frac{x^{3}}{x^{2} - x}
\]
\[
\lim_{x \to \tfrac{\pi}{2}} \int_{0}^{\infty} \frac{\cos ^{2}x}{\sin ^{3}x - 1} \, \mathrm{d}x
\]

\begin{flushright}
\subsection{نماد های ترکیبی}
   مثلا... 
   میانگین :
   \[
   \bar{x} = \frac{\Sigma_{0}^{n} x_{i}}{n}
   \]
\end{flushright}

\end{document}


