\documentclass{book}

\usepackage{setspace}
\usepackage{paralist}
\usepackage{multicol}
\usepackage{hyperref}
\usepackage{shapepar}
\usepackage{amssymb}
\usepackage{breqn}
\usepackage{amsfonts}
\usepackage{amsmath, amsthm}
\usepackage{xepersian}

\settextfont{Traditional Arabic}
\setcounter{tocdepth}{4}
\setcounter{secnumdepth}{4}
\setdigitfont{Traditional Arabic}

\newtheorem{thm}{قضیه}[chapter]
\newtheorem{Def}{تعریف}[section]

\newcommand{\dd}{\, \mathrm{d}}
\newcommand{\mian}[2]{\frac{#1_1 + #1_2 + \dots + #1_{#2}}{n}}

\defpersianfont\BNaz{B NAZANIN.TTF}
\deflatinfont\Time{TIMES.TTF}

\huge{\title{\lr{$4th$ $File$}}}
\author{علی}

\begin{document}
	\maketitle
	%...
	%\doublespacing
	\tableofcontents
	\singlespacing
	%...
	\Time
\chapter{\lr{$table$}}	
\BNaz
بررسی ساختن جدول:
\begin{table}[h!]

\begin{center}
 \begin{tabular}{|r|p{2cm}||l|}
	\hline
	\textbf{نام} &
	\centering
	 نام خانوادگی & قد \\
	\hline
شماره تماس &
\centering
 رشته & تاهل\\
	\hline
 \end{tabular}
\end{center}
\caption{استخدام}
\label{Tab1}
\end{table}
\\
	در جدول \ref{Tab1} دیدیم که ...
\end{document}