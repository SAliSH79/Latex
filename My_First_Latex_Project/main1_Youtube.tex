%%%%%%%%% ALi SH %%%%%%%%%%%%%%%%%%

\documentclass[12pt]{article}
\usepackage[utf8]{inputenc}
\usepackage{amsmath}
\usepackage{graphicx}

\title{Hello}

\author{ALI H}

\date
% Preamble

\begin{document}

\maketitle

\section{Introduction}
hello guys

\subsection{Formulas}
\begin{enumerate}
\item 
Let`s say $e^{i\pi}$ is \textbf{equal} to -1. %In line Formula
\item 
But Also we can do: 
$$ e = \lim_{n \to \infty } \left ( 1 + \frac{1}{n} \right ) ^n = \lim_{n \to \infty } \frac{n}{\sqrt[n]{n!}}$$
\item
We can do \textit{another}:
$$ e = \sum_{0}^{\infty} \frac{1}{n!}$$
\end{enumerate}

\subsection{More Formulas}
\hspace{0.61cm}Vectors:

$$\vec{V} =<V1, V2, V3>$$

Matrices:
$$
\begin{bmatrix}
    1 & 2 & 3\\
    4 & 5 & 6
\end{bmatrix}
$$

\subsection{Images}

\includegraphics[scale = 0.1]{LOS2.jpg}
\\

\section{Tables and Theorems}
\begin{equation}
    \tan x \cdot \cot x = 1
    \label{tan}
\end{equation}
equation \eqref{tan} is really useful in advanced mathematics.

\begin{equation}
    \begin{split}
        e & = 2.37 \\
        \text{ali and Reza} & = 2.37
        \label{e}
    \end{split}
\end{equation}
in equation \eqref{e} we can witness that \dots


\end{document}
